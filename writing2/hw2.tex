\documentclass[journal, letterpaper, draftclsnofoot, onecolumn, 10pt]{journal}

\usepackage{graphicx}
\usepackage{amssymb}
\usepackage{amsmath}
\usepackage{amsthm}

\usepackage{alltt}
\usepackage{float}
\usepackage{color}
\usepackage{url}
\usepackage{listings}

\usepackage{balance}
\usepackage[TABBOTCAP, tight]{subfigure}
\usepackage{enumitem}
\usepackage{pstricks, pst-node}
\usepackage{placeins}
\usepackage{geometry}
\usepackage{hyperref}
\geometry{textheight=8.5in, textwidth=6in}

\lstset{
  language=C,                % choose the language of the code
  numbers=left,                   % where to put the line-numbers
  stepnumber=1,                   % the step between two line-numbers.
  numbersep=5pt,                  % how far the line-numbers are from the code
  backgroundcolor=\color{white},  % choose the background color. You must add \usepackage{color}
  showspaces=false,               % show spaces adding particular underscores
  showstringspaces=false,         % underline spaces within strings
  showtabs=false,                 % show tabs within strings adding particular underscores
  tabsize=8,                      % sets default tabsize to 2 spaces
  captionpos=b,                   % sets the caption-position to bottom
  breaklines=true,                % sets automatic line breaking
  breakatwhitespace=true,         % sets if automatic breaks should only happen at whitespace
  title=\lstname,                 % show the filename of files included with \lstinputlisting;
}

%random comment

\newcommand{\cred}[1]{{\color{red}#1}}
\newcommand{\cblue}[1]{{\color{blue}#1}}

\newcommand{\toc}{\tableofcontents}


\def\name{Arthur Shing}

%% The following metadata will show up in the PDF properties
\hypersetup{
   colorlinks = true,
   urlcolor = black,
   linkcolor = black,
   pdfauthor = {\name},
   pdfkeywords = {cs444 ``operating systems'' files filesystem I/O},
   pdftitle = {CS 444 Writing Assignment 1},
   pdfsubject = {CS 444 Writing Assignment 1},
   pdfpagemode = UseNone
}

\parindent = 0.0 in
\parskip = 0.1 in


\begin{document}
\title{Project 1: Getting Acquainted}
\author{Arthur Shing}


\begin{titlepage}
    \pagenumbering{gobble}
    \centering
    \maketitle
    \large Spring 2017



\end{titlepage}
\pagenumbering{arabic}
\tableofcontents
\clearpage


\section{Processes, Threads and CPU Scheduling}

Processes, threads, and CPU Scheduling are parts of an operating system that define how the operating system functions. Different operating systems have different scheduling algorithms, and processes running on one operating system may behave differently on another operating system. In the following sections, I will examine the similarities and differences of these three components in Linux, FreeBSD, and Windows.

\subsection{Processes}

A process is often defined as a program in execution, or as an instance of running programs. Processes do not only include the program, but also the resources that are needed to run the program. The specific components of a process differ across operating systems; however, there are some properties of a process that are apparent across all operating systems.

Processes are represented by a large data structure called a Process Control Block. In Linux, processes are represented by the task\_struct data structure. Each new process is appended to a task vector, which has by default can hold up to a limit of 512 processes. In Windows, processes are defined in the EPROCESS block.  Linux, FreeBSD, and Windows all contain similar components in their process control blocks. They all include things such as a unique PID, the execution state, virtual address space, program code, program data, scheduling information, and system resources.

In Linux and FreeBSD, processes are created using the fork() and exec() calls, which creates a child process. This clones the current process, so that the child and parent process shares important resources. In Windows, the functional equivalent is CreateProcess(), which creates a new process but does not initialize a parent child relationship like Linux does. Because of this, child processes in Windows only store the PID of their parent process and no other information, and can continue running when a parent process is closed. In Linux, child processes tend to exit when the parent exits.

\section{Input/Output}

I/O is shorthand for input/output, and is used by computer scientists to describe the interaction between a computer system and anything outside of the computer system, such as users and external devices.
In the following sections, I will discuss I/O functionality in Windows, FreeBSD, and Linux systems.

\subsection{Functionality}

In terms of functionality, I/O devices are scheduled in Linux with different built in functions. Some scheduling algorithms provided in the kernel inclkude the Linus Elevator, Deadline algorithm, No-op, and CFQ (Completely Fair Queueing).
The Linus Elevator acts as it seems like it should - an elevator. It does merging and sorting, and sends requests to the executing queue in the same way an elevator would. The deadline algorithm acts similarly, but keeps requests in a separate FIFO queue to
prevent starvation. The provided No-op scheduler acts as a pure FIFO, and neither merges nor sorts. As a result it is not optimal for dealing with latency and overhead costs of moving the read/write head. Linux's default scheduler, the CFQ scheduler, aims to schedule requests as fairly as possible.

The Linux kernel provides data structures for I/O scheduling. For instance, the list_head struct acts as a functioning list for queueing.

\subsection{Windows & FreeBSD}



\FloatBarrier
\end{document}
